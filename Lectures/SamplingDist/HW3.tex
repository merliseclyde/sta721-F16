\documentclass[12pt]{article}
\usepackage{fullpage}
\usepackage{amsmath,amssymb,array,comment,eucal}
\pagestyle{empty}
\input{../macros}
\begin{document}
{\bf STA721 \hfill Homework 3}

\vspace{.5in}


\begin{enumerate}
\item Condsider the linear model $\Y \sim \N(\mub, \sigma^2 \I_n)$
  with  $\mub = \one \beta_0 + \X \b$ and $\X$ a full rank matrix with rank $p$
  \begin{enumerate}
  \item 

    Show that the projection, $\P$, on the column space spanned by
    $\one$ and $\X$ may be written as
    $\P = \P_{\one} + \P_{\X - \one \xbar^T}$.  Show that diagonal
    elements are
    $$h_{ii} = \frac{1}{n} + (\x_i-\xbar)^T\left((\X- \one \xbar^T)^T(\X - \one
    \xbar^T)\right)^{-1}(\x_i - \xbar)$$
    (Recall all vectors are column vectors).  The $h_{ii}$ are known
    as the leverage values.
\item Find the sampling distribution of $\hat{\mu}_i$ (the mean
  of  $Y_i$ at $\x_i^T$ as a function of $h_{ii}$ and provide an
  expression for a 95\% confidence interval.   For what values of $\x$
  will the interval be the narrowest? Explain.

\item Given $\sigma^2$, find the distribution of $\e_i$ as a function
  of $h_{ii}$.  Explain (rigorously) why $\e_i$ unconditional on
    $\sigma^2$ does not have a student $t$ distribution with
    $n - p - 1$ degrees of freedom.
\end{enumerate}
\item  Refer to the Prostate data  from {\tt library(lasso2); data(Prostate)}
\begin{enumerate}

\item Fit a linear model using {\tt lcavol} (log
  cancer volume) as the response and include all covariates.
  Construct 95\% confidence intervals for 
  each coefficient and provide a meaningful interpretations for
  changes in the  cancer volume  ( not log cancer volume) include any
  units etc in your interpretation.   See Wakefield page 1.3.1 for
  details on variables.  Note ``a 1 unit'' change may or may not be
  meaningful for interpretation.   
\item Fit the regression model with
  response {\tt lcavol}, and variables {\tt svi} and {\tt lpsa} as
  predictors.  Plot the cancer volume versus PSA on the log scale. Add
  the fitted regression function for svi = 1 and svi = 0, with
  lines representing the (pointwise) 95\%  confidence intervals for each. 
\end{enumerate}

\end{enumerate}
\end{document}