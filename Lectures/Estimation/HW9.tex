\documentclass[12pt]{article}
\usepackage{fullpage, amssymb,amsmath}
\pagestyle{empty}
\input{../macros}
\begin{document}
\begin{center}
{\bf STA721  Homework 9} \\
\end{center}
\vspace{.5in}

\noindent
Assume  the model
\begin{equation}
  \label{eq:true}
\Y = \one \alpha + \X \betav + \eps 
\end{equation}
where $\X$ is $n \times p$ and
full column rank for the problems below.  Without loss, you may assume
that $\one^T\X = \zero_p$ for below.


\begin{enumerate}

 \item What is  $E_{\Y \mid \b, \phi}[\|\hat{\b} - \b)\|^2]$,
  the expected MSE for OLS under the full model?

\item  Find  $\btilde = \E_{\b \mid \Y}[\b \mid \Y, g]$, the
  posterior mean under the Zellner $g$-prior:
  \begin{align*}
  p(\alpha, \phi) & \propto \phi^{-1}     \\
  \b \mid \phi, g & \sim N(\zero_p, \frac{g}{\phi} (\X^T \X)^{-1})
  \end{align*} as a function of the MLE $\bhat$.

\item  Find the {\it sampling
  distribution} of $\btilde$. (i.e as a function of $\Y$ given
parameters, what is the  distribution of $\btilde$?)
  
\item  Is the posterior mean $\btilde$ 
   unbiased for estimating $\b$?  If not, what is the bias?  (again the
   expectation is with respect to the distribution for $\Y \mid \b, \phi$

 \item Find $\E_{\Y \mid \b, \phi}[\| \btilde - \b \|^2]$ assuming
   model (1) and express as a function of $g$, $\|\b\|^2$ and
   expected MSE for
   OLS (if possible).  This expectation should be taken with respect
   to the sampling distribution of $\Y$ not the posterior distribution
   of $\b$.


 \item The Gauss-Markov Theorem showed that out of the class of
   unbiased linear estimators, the MLE has the smallest variance.  If
   we use the posterior mean above, can the posterior mean have a
   smaller loss than the MLE for estimating $\b$?  Can it be much
   worse? Explain.  (Make a plot to illustrate with $g/(1+g)$ on the
   x-axis and MSE on the y-axis); add curves for 1) the sum of the
   squared bias terms and 2) the variance term (the part from the
   trace) as a additional lines using different line types and a
   legend.  (you may need to assume or fix values for some quantities
   that go into the loss, if so how sensitive are the
   plots/conclusions to those assumptions?).

 \item Can you find a value of $g$ to minimize the Expected MSE with
   the Bayes estimator?  If so, what is it?  Add this point to you
   graph above.   Simplify as much as
   possible.  With this $g$ will the Expected MSE with the Bayes
   estimator always be smaller than that with the MLE/OLS?  If this depends
   on the parameters, describe how you could estimate it.



 \end{enumerate}

\end{document}



%\item Christensen Exercises 4.1, 4.2, 4.3,  4.6

%\item Refer to the  Meadowfoam data from class.
 
%\begin{enumerate}
% \item 
%Show
%that model that allows a different mean for every combination of Time
%and Intensity is equivalent to the Two-way AOV 
%with interaction model of Sec 7.2.  

%\item Fit the two-way AOV model (with interactions) in R and obtain an
%ANOVA table.  Do there appear to be significant interactions between
%Time and Intensity?

\item Show that the mean vector for the regression model discussed in
  class with Intensity   now as a numeric variable and allowing for
  different intercepts and   slopes for the two Time periods 
 is contained in the space spanned by the columns of the design
  matrix for the 2-way ANOVA model with interaction (see Page 163).

\item Carry out the Extra-Sum of Squares F-test for Lack of Fit of the
  regression model above (see Chapter 6).  What  do you conclude?

\item Which model do you recommend for the data (of the above or those
  from class)?  Using this model, write a brief paragraph summarizing
  the effects of light intensity and photoperiod timing on flower
  production. Make sure that any point estimates are accompanied by
  appropriate interval estimates.



13.2, 13.6, 13.7, 13.8, 13.9 Exercise 11.23 from Statistical Sleuth
