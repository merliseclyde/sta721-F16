\documentclass{article}


\usepackage{graphicx, fullpage}
\usepackage{amsmath,amssymb,array,comment,eucal}
\input{../macros}
\usepackage{verbatim}
\begin{document}
\begin{center}
  STA 721 HW 6
\end{center}


\begin{enumerate}
\item If $\Y \sim \N(\X\b, \sigma^2 \I_n)$, show that the likelihood
  function for $\b, \phi$ where $\phi = 1/\sigma^2$ can be written as 
 $$\cL(\beta, \phi) \propto \phi^{n/2} e^{- \phi \frac{\SSE}{2}}
e^{-\frac{\phi}{2} (\b - \bhat)^T(\X^T\X) (\b - \bhat)}$$.   Do you
need to consider a Jacobian term for a change of variables?  (explain)

\item Consider the prior data dependent prior $\b \mid \phi \sim N(\bhat, \sigma^2 n
  (\X^T\X)^{-1})$ and   $\phi \sim G((n + 2)/(2n), \SSE/(2n)$
where $\bhat$ is the MLE of $\b$, $\X$ is $n \times p$ and rank $p$
and $\SSE$ is the residual sum of squares.
  \begin{enumerate}
  \item  Find the prior mean of $\sigma$ and $\sigma^2$.
  \item  Using the likelihood above, find the conditional posterior distribution of $\b$ given
    $\phi$ and the marginal posterior distribution for $\phi$,
    simplifying as much as possible.  What
    is the posterior mean for $\b$ and $\sigma^2$?   
  \item Find the marginal distribution of $\beta_j$.
  \item Suppose that $\beta_j \mid \Y, \phi$ are independent.  What
    does that imply about $\X$?  Will the $\beta_j$ be independent
    after marginalizing $\phi$?
  \end{enumerate}
\end{enumerate}

\end{document}
