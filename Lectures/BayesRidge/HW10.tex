\documentclass{article}


\usepackage{graphicx, fullpage}
\usepackage{amsmath,amssymb,array,comment,eucal}
\input{../macros}
\usepackage{verbatim}

\begin{document}
\begin{center}
  STA 721 HW 10
\end{center}

\begin{enumerate}
\item With the model from class and sufficient statistics, derive  the
  full conditional distributions  for  $\alpha$,   $\gamma$ $\kappa_j$
  and $\phi$ assuming 
  \begin{align}
    p(\alpha, \phi) & \propto 1/\phi \\
    \gamma_j \mid \kappa_j, \phi, \alpha & \simiid \N(0, \frac{1}{\phi
                                           \kappa_j}) \\
   \kappa_j & \iid G(1/2, 1/2)
  \end{align}
(You should have a name, and 
  expressions for all hyperparameters)
\item Modify you Gamma prior on $\kappa_i$ to capture the  desired features
  based on $l_i$. 

\item Find the updated full conditionals based on your choice above.
  Do you need to update all of the full conditionals?  Explain.

\item Implement your models in R or JAGS  (see earlier JAGS code as a
  starting point) and apply this to the {\tt longley} data.
 How do your results compare to classical ridge?  Include plots of the
 posterior distributions of coefficients, plus means and credible
 intervals, as well as plots of the distributions of the $\kappa$'s.
 How sensitive to the results to the prior assumptions?  How do the
 estimates of $\kappa_i$ compare to the best GCV estimate from class?

\item Explain the compuational advantage of using the canonical parameterization
  in  MCMC.
\end{enumerate}
\end{document}
