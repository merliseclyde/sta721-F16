\documentclass[12pt]{article}
\usepackage{fullpage,amssymb,amsmath, url}
\pagestyle{empty}
\input{../macros}
\begin{document}
{\bf STA721}
\vspace{.1in}
\begin{center}
{\large \bf Homework 17} \\
\end{center}

The Wage data in {\tt library(ISLR); data(Wage)} includes a set of
variables thought to be associated with wages  (age, education level,
year, etc) for a group of males from the Atlantic region of the US. 

\begin{itemize}
\item Split the data into a training set and validation set (say 80\%
  training and 20\% for validation).  (keep your random seed in case
  you need to reproduce).   Using
  the training data, explore adding additional variables to the 
  generalized additive model for wage to construct a model for the
  data.  (see the documentation for {\tt gam} or {\tt bam} in
 {\tt library(mgcv)} for options and details on other smoothing
  splines/priors, i.e.  you may wish to fix $k$ to prevent some
  over-fitting). Narrow down your models to what you might consider
  your top 5.   
\item Using several of your ``best'' models, calculate the likelihood
  ratio for comparing log(wage) to wage as the best response for
  normality.  (e.g. BoxCox but with just two choices).  Which
  transformation is better?  does that agree with residual plots?

\item Use your models to predict wages on the validation data.  Using
  boxplots of predicted residuals and predicted MSE, which model(s)
  are best for predicting Wages?  Does this agree with your findings
  on the training data?  

\item Using R Markdown or KnitR, provide a write up of up to 3 pages
  with key figures that introduces the problem, your methods and findings
  (i.e. key interactions, nonlinearities or other relationships with figures to
  support key findings) that describe the relationship of wages with
  the other variables.  You may   include all of your R code as an appendix but
  code should not appear in the main document.
\end{itemize}

\end{document}