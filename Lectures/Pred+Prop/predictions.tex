%\documentclass{beamer}
\documentclass[handout]{beamer}
%\usepackage[dvips]{color}
\usepackage{graphicx}
\usepackage{amsmath,amssymb,array,comment,eucal}
\input{../macros}
\usepackage{verbatim}

\usetheme{Warsaw}
\title{Predictive Distributions \& Properties of MLES}
\subtitle{Merlise Clyde}
\author{STA721 Linear Models}
\institute{Duke University}
\date{September 15, 2016}
\logo{duke.eps}

\begin{document}
\maketitle

\begin{frame}\frametitle{Outline}
Topics
  \begin{itemize}
  \item Predictive Distributions
  \item OLS/MLES Unbiased Estimation
  \item Gauss-Markov Theorem
  \end{itemize}


Readings: Christensen Chapter 2,  Chapter 6.3, ( Appendix A, and
Appendix B as needed)
\end{frame}

\begin{frame} \frametitle{Prediction}

  \begin{itemize}
  \item Predict $Y_*$ at $\x_*^T$   (could be new point or existing
    point) $\Y_* = \x_*^T \b + \epsilon_*$ \pause
 \item $\E[Y_* \mid \x_*] = \x_*^{T} \b = \mu_*$ minimizes squared
   error loss for predicting $Y_*$ at $\X_*^T$ \pause
   \begin{eqnarray*}
     \E[Y_* - f(\x_*)]^2 & = & \E[Y_* - \mu_* + \mu_* - f(x_*)]^2
                               \pause \\
 & = & E[Y_* - \mu_*]^2 + E[\mu_* - f(x_*)]^2 + \\  &  & 2\E[( Y_* -
       \mu_*)(\mu_* - f(x_*))] \pause \\
 & \geq & E[Y_* - \mu_*]^2   \end{eqnarray*} \pause
Crossproduct term is 0:  
$$\E[\E[( Y_* -
       \mu_x)(\mu_* - f(x_*))\mid \x_*]] = \E[0 \cdot (\mu_* -
       f(x_*))]$$
\item equality if $f(x) = E[Y_* \mid \x_*]$,  the ``best'' predictor
  of $Y_*$ \pause
\item MLE of $\mu_*$ is $\x_*^{T} \bhat  = \hat{Y}_*$ (is this
  unique?) \pause
\item OLS Best Linear predictor of $\Y_*$
\item Under joint Normality of $\Y, \X$  Best Predictor
  \end{itemize}
\end{frame}

\begin{frame}\frametitle{Predictive Distribution}
Look at  
 $$Y_* -\hat{Y}_*  = \x^{*T} \b - \x_*^T\bhat + \epsilon_*$$ \pause

 $$\var(Y -\hat{Y})  = \var(\x_*^{T} \b - \x_*^T\bhat) +
 \var(\epsilon_*)$$ \pause

\vfill
Two Sources of variation:
\begin{itemize}
\item Variation of estimator around true regression \pause
\item Variation of error around true regression \pause
\end{itemize}
\end{frame}
\begin{frame}\frametitle{Distribution}
Distribution of 
$$ \frac{Y_* - \x_*^T\bhat}{
\sqrt{\MSE (1 + \x_*(\X^T\X)^{-1}\x_*^T)}} \sim t(n - p - 1, 0, 1)
$$  
\vfill

$(1 - \alpha) 100$ \% Prediction Interval 
 $$\x_*^T \bhat \pm  t_{\alpha/2} \sqrt{\MSE (1 + \x_*(\X^T\X)^{-1}\x_*^T)}$$
\end{frame}

\begin{frame} \frametitle{Models \& MLEs}
  \begin{itemize}
  \item   $\Y \sim \N(\mub, \sigma^2 \I_n)$ with $\mub \in C(\X)
    \Leftrightarrow \mub = \X \b$  \pause
  \item  Maximum Likelihood Estimator (MLE) of $\mub$ is
    $\P_\X \Y$  \pause
\item $\P_\X$ is the orthogonal projection operator on the column
  space of $\X$; e.g. $\X$ full rank $\P_\X = \X (\X^T\X)^{-1} \X^T$
  \pause
\item If $\X^T\X$ is not invertible use a generalized inverse
\end{itemize}
A generalize inverse of $\A$: $\A^{-}$ satisfies   $\A \A^- \A = \A$
\begin{lemma}[B.43]
  If $\, \G$ and $\H$ are generalized inverses of $(\X^T\X)$ then
  \begin{enumerate}
  \item $\X \G \X^T \X = \X \H \X^T \X = \X$
  \item $\X \G \X^T = \X \H \X^T$
  \end{enumerate}
\end{lemma} \pause
$\P_\X = \X (\X^T\X)^{-} \X^T$ is the orthogonal projection operator
onto $C(\X)$   (does not depend on choice of generalized inverse!)
[See proof in Theorem B.44]
\end{frame}
\begin{frame}
  \frametitle{Generalize Inverses}

A generalize inverse of $\A$: $\A^{-}$ satisfies   $\A \A^- \A = \A$


Special Case: Moore-Penrose Generalized Inverse  \pause


\begin{itemize}
\item Decompose symmetric $\A = \U \Lambdab \U^T$  \pause
\item $\A^-_{MP} = \U \Lambdab^- \U^T$  \pause
\item $\Lambdab^-$ is diagonal with $$ \lambda_i^- = \left\{
    \begin{array}{l}
   1/\lambda_i \text{ if } \lambda_i \neq 0 \\
   0 \quad \, \text{  if } \lambda_i = 0
    \end{array}
\right.$$  \pause
\item Symmetric  $\A^-_{MP} = (\A^-_{MP})^T $  \pause
\item Reflexive  $\A^-_{MP}\A \A^-_{MP} = \A^-_{MP} $  \pause
\end{itemize}

If $\P$ is an orthogonal projection matrix, the generalized inverse of
$\P$, $\P^- = \P$ 


\end{frame}
\begin{frame}
  \frametitle{MLE of $\b$}
  \begin{eqnarray*}
    \P_\X \Y & = & \X \bhat  \\
\X(\X^T\X)^- \X^T\Y & =
& \X \bhat 
  \end{eqnarray*}  \pause
  \begin{itemize}
\item MLE of $\b$ iff $\P_\X \Y = \X \bhat$
\item If $\X^T\X$ is invertible, then
$$\bhat = (\X^T\X)^{-1} \X^T\Y$$ and is unique  \pause

\item But if $\X^T\X$ is  not invertible, $$\bhat = (\X^T\X)^{-} \X^T\Y$$ is
one solution which depends on choice of generalized inverse
 \pause
  \end{itemize}

What can we estimate uniquely?
\end{frame}
\begin{frame}
  \frametitle{Identifiability}
$\Y \sim \N(\mub, \sigma^2 \I)$  \pause
  \begin{itemize}
  \item Distribution of $\Y$ determined by $\mub$ and $\sigma^2$  \pause
\item $\mub = \X\b = \mu(\b)$  \pause
\end{itemize}
  \begin{block}{Identifiability}
$\b$ and $\sigma^2$ are identifiable if distribution of $\Y$, $f_\Y(\y;
\b_1, \sigma^2_1) = f_\Y(\y;
\b_2, \sigma^2_2)$ implies that $(\b_1, \sigma^2_1)^T =  (\b_2, \sigma^2_2)^T$
  \end{block}  \pause
For linear models, equivalent definition is that $\b$ is identifiable
if for any $\b_1$ and $\b_2$ $\mu(\b_1) = \mu(\b_2)$ implies that
$\b_1 = \b_2$.  If $ r(\X) = p$ then $\b$ is identifiable
 \pause
\vspace{12pt}
If $\X$ is not full rank, there exists $\b_1 \neq \b_2$, but $\X\b_1 =
\X\b_2$ and hence $\b$ is not identifiable
\end{frame}


\begin{frame}
\frametitle{Non-Identifiable }
  Recall the One-way ANOVA model \pause
$$\mu_{ij} = \mu + \tau_j \qquad \mub = (
    \mu_{11}, \ldots,\mu_{n_1 1},\mu_{12},\ldots, \mu_{n_2,2},\ldots, \mu_{1J},
\ldots,
\mu_{n_J J})^T $$ \pause
\begin{itemize}
\item Let $\b_{1} = (\mu, \tau_1, \ldots, \tau_J)^T$ \pause
\item Let $\b_{2} = (\mu - 42, \tau_1 + 42, \ldots, \tau_J + 42)^T$ \pause
\item Then $\mub_{1} = \mub_{2}$ even though  $\b_1 \neq \b_2$ \pause
\item $\b$ is not identifiable \pause
\item yet  $\mub$ is identifiable, where $\mub = \X \b$  (a linear
  combination of $\b$) 
\end{itemize}
\end{frame}

\begin{frame}
  \frametitle{Identifiability and Estimability}
  \begin{theorem}
    A function $g(\b)$ is identifiable if and only if $g(\b)$ is a
    function of $\mu(\b)$
  \end{theorem} \pause
In linear models,  historical focus on linear functions.  Identifiable linear
functions are called {\it estimable} functions \pause

\begin{definition}
  A vector valued function $\Lambdab \b$ is {\it estimable} if $\Lambdab \b
  = \A \X \b$ for some matrix $\A$ \pause
\end{definition} 
Equivalently 

\begin{definition}
  A vector valued function $\Lambdab \b$ is {\it estimable} if it has
  an unbiased linear estimator,   i.e. there exists an $\A$ such that
  $\E(\A \Y) = \Lambdab \b$ for all $\b$
\end{definition}  

\end{frame}
\begin{frame}
  \frametitle{Estimability}
  Work with scalar functions $\psi = \lambdab^T \b$
  \begin{theorem}
    The function $\psi = \lambdab^T \beta$ is estimable if and only if
    $\lambdab^T$ is a linear combination of the rows of
    $\X$. i.e. there exists $\a^T$ such that  $\lambdab^T = \a^T \X$
  \end{theorem} \pause
  \begin{proof}
   The function $\psi= \lambdab^T\b$ is estimable if there exists an $\a^T$ such that
    $\E[\a^T\Y] = \lambdab^T\b$ \pause

    \begin{eqnarray*}
      \E[\a^T\Y] & =  &\a^T \E[\Y] \pause\\
                 & = & \a^T \X \b \pause \\
  & =& \lambdab^T \b  \pause
    \end{eqnarray*}
if and only if $\lambdab^T = \a^T \X$ for all $\b$
  \end{proof}
\end{frame}
\begin{frame}
  \frametitle{Estimability of Individual $\beta_j$}
  \begin{block}{Proposition}
For    $$\mub = \X \b = \sum_j \X_j \beta_j$$
$\beta_j$ is not identifiable  if and only if there exists $\alpha_j$
such that $\X_j = \sum_{i \neq j} \X_i \alpha_i$
  \end{block} \pause
One-way Anova Model:
$$Y_{ij} = \mu + \tau_j + \epsilon_{ij}$$
  $$ \mub =  \left[
    \begin{array}{lllll}
\one_{n_1} & \one_{n_1} & \zero_{n_1} &  \ldots & \zero_{n_1} \\
\one_{n_2} & \zero_{n_2} & \one_{n_2} &  \ldots & \zero_{n_2} \\
\vdots & \vdots & \ddots & \vdots \\
\one_{n_J} & \zero_{n_J} & \zero_{n_J} &  \ldots & \one_{n_J} \\
    \end{array} \right]
 \left(   \begin{array}{l}
      \mu \\
      \tau_1 \\
   \tau_2 \\
 \vdots \\
\tau_J
    \end{array} \right)
$$\pause
Are any parameters $\mu$ or $\tau_j$ identifiable?
\end{frame}

\begin{frame}
  \frametitle{Gauss-Markov Theorem}
  \begin{theorem}

  Under the assumptions:
  \begin{eqnarray*}
    \E[\Y] & = & \mub \pause \\
    \Cov(\Y) & = &\sigma^2 \I_n \pause
  \end{eqnarray*}
every estimable function $\psi = \lambdab^T\b$ has a unique unbiased
linear estimator $\hat{\psi}$ which has minimum variance in the class
of all unbiased linear estimators. \pause  $\hat{\psi} = \lambdab^T\bhat$
where $\bhat$ is any set of ordinary least squares
estimators.
    
  \end{theorem}
\end{frame}
\begin{frame}
  \frametitle{Unique Unbiased Estimator}
  \begin{block}{Lemma}
    \begin{itemize}
    \item 
    If   $\psi = \lambdab^T \b$ is estimable, there exists a unique
    linear unbiased estimator of $\psi = \a^{*T}\Y$ with $a^* \in
    C(\X)$. \pause 
\item If $\a^T\Y$ is any unbiased linear estimator of $\psi$
    then $a^*$ is the projection of $\a$ onto $C(\X)$, i.e. $\a^* =
    \P_\X \a$.
    \end{itemize}

  \end{block} 
      \end{frame}
  \begin{frame}
 \frametitle{Unique Unbiased Estimator}
  \begin{block}{Proof}
    \begin{itemize}
    \item     Since $\psi$ is estimable, there exists an $\a \in \bbR^n$ for
    which $\E[\a^T\Y] = \lambdab^T\b = \psi$ \pause
\item Let $\a = \a^* + \u$ where $\a^* \in C(\X)$ and $\u \in
  C(\X)^\perp$ \pause
\item Then
  \begin{eqnarray*}
\psi  =   \E[\a^T\Y] &=& \E[\a^{*T}\Y] + \alert<3-5>{\E[\u^T\Y]} \pause\\    
     & = & \E[\a^{*T} \Y] + \alert<4>{0} \\
  &  & \\
\E[\u^T\Y] & = & \u^T\X \b
  \end{eqnarray*} 
since $\u \perp C(\X)$ (i.e. $\u \in C(\X)^\perp$) $\E[\u^T\Y] = 0$ \pause

\item Thus $\a^{*T} \Y$ is also an unbiased linear estimator of $\psi$ with
  $\a^* \in C(\X)$
    \end{itemize}

  \end{block}
\end{frame}
\begin{frame}
  \frametitle{Uniqueness}
  \begin{proof}
  Suppose that there is another $\v \in C(\X)$ such that $\E[\v^T\Y] =
  \psi$. Then for all $\b$ \pause
  \begin{eqnarray*}
    0 & = & \E[\a^{*T}\Y] - \E[\v^T\Y]  \pause \\
      & = & (\a^* - \v)^T \X \b \pause\\
\text{So  } (\a^* - \v)^T\X & = &  0 \quad \text{ for all } \b \pause
  \end{eqnarray*}

\begin{itemize}
\item Implies $(\a^* - \v) \in C(\X)^\perp$ \pause
\item but by assumption $(\a^* - \v) \in C(\X)$ \pause ($C(\X)$ is a
  vector space) \pause
\item the only vector in BOTH is $\zero$, so $\a^* = \v$ \pause
\end{itemize}
Therefore $\a^{*T}\Y$ is the unique linear unbiased estimator of $\psi$
with $\a^* \in C(\X)$.
  \end{proof}

\end{frame}
\begin{frame} \frametitle{Proof of Minimium Variance}
 \begin{block}{}
   \begin{itemize}
    \item
    Let $\a^{*T}\Y$ be the unique unbiased linear estimator of $\psi$
    with $\a^* \in C(\X)$. \pause
\item Let $\a^T\Y$ be any unbiased estimate of $\psi$; $\a = \a^* +
  \u$ with $\a^* \in C(\X)$ and $\u \in C(\X)^\perp$ \pause
  \begin{eqnarray*}
    \Var(\a^T\Y) & = & \a^T\Cov(\Y)\a  \pause\\
 & = & \sigma^2 \| \a \|^2  \pause \\
& = & \sigma^2 (\| \a^* \|^2 + \|\u\|^2 + 2 \a^{*T} \u) \pause\\
& = & \sigma^2 (\| \a^* \|^2 + \|\u\|^2) + 0 \pause\\
& = & \Var(\a^{*T}\Y) + \sigma^2 \|\u\|^2 \pause \\
& \geq & \Var(\a^{*T}\Y) \pause
  \end{eqnarray*}
with equality if and only if $\a = \a^*$ \pause
\end{itemize}
Hence $\a^{*T}\Y$ is the unique linear unbiased estimator of $\psi$
with minimum variance \pause {\color{blue}''BLUE'' = Best Linear Unbiased Estimator}
  \end{block}
\end{frame}
\begin{frame}
  \frametitle{Continued}
  \begin{proof}
  Show that $\hat{\psi} = \a^{*T}\Y = \lambdab^T\bhat$ \pause

  Since $\a^* \in C(\X)$ we have $\a^*  =  \P_\X \a^*$  \pause
    \begin{eqnarray*}
\a^{*T}\Y & = &  \a^{*T} \P_X^T \Y \pause \\
         & = & \a^{*T}\P_x\Y \pause \\
         & = & \a^{*T} \X \bhat \pause \\
         & = & \lambdab^T \bhat  \pause
    \end{eqnarray*}
for $\lambdab^T = \a^{*T}\X$
  \end{proof}
\end{frame}
\begin{frame}
  \frametitle{MVUE}
  \begin{itemize}
  \item Gauss-Markov Theorem says that OLS has minimum variance in the
    class of all Linear Unbiased estimators \pause
\item Requires just first and second moments \pause
\item Additional assumption of normality,  OLS = MLEs have
  minimum variance out of \alert<3>{ALL}  unbiased estimators; not
  just linear estimators \pause (requires Completeness and
Rao-Blackwell Theorem - next semester) \pause
\item Mean Squared Error for estimator $g(\Y)$ of $\lambdab^T\b$ is
$$\E[g(\Y) - \lambdab^T\b]^2 = \Var(g(\Y)) + \text{Bias}^2(g(\Y))$$
where Bias = $\E[g(\Y)] - \lambdab^T\b$  \pause
\item Bias vs Variance tradeoff \pause
\item Can have smaller MSE if we allow some Bias! 
  \end{itemize}

\end{frame}




\end{document}

