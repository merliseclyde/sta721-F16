\documentclass{article}

\usepackage{fullpage}
\usepackage{amsmath,amssymb,array,comment,eucal}
\pagestyle{empty}
\input{../macros}
\begin{document}
{\bf STA721 \hfill Homework 4}

\vspace{.5in}
\begin{enumerate}
\item Add 95\% prediction intervals to your plot from HW3 for the
  Prostate data using a different linetype and color.  Explain why the
  prediction intervals are wider than the confidence intervals for
  $\hat{\mub}$.  (See the function {\tt predict()} in \R.  Please
  label all axes with units and informative names, add a legend to
  explain the multiple lines, and a caption).    (Post your
  code/example from HW 3 on Piazza for participation points and vote
  on the best solution!)

\item Consider the linear model  model $\Y \sim \N(\mub, \sigma^2 \I_n)$
  with  $\mub =  \X \b$ and $\X$ a full rank matrix with
  rank $p$.  For a new observation $Y_*$ at $\x_*$ with $Y_* =
  \x_*^T\b + \epsilon_*$ and $\epsilon_*$ independent of $\eps$,  consider the
  predicted residual  $Y_* - \x^T_* \bhat$ where $\bhat$ is the MLE
  using data $\Y$.   
  \begin{enumerate}
  \item Find the distribution of the predicted residual $Y_* - \x^T_*
    \bhat$ given $\b$ and  $\sigma^2$. 
\item Show that the standardized predicted residual (center so that
  the mean is 0 and and scale (sd) is 
  1 with $\sigma^2$ replaced by the usual unbiased estimate
  $\hat{\sigma}^2 = \Y^T(\I-\P_{\X})\Y/ (n - p - 1)$ has a student $t$
  distribution.  What are the degrees of freedom?   
  \end{enumerate}
  \item   Consider the linear model $\Y = \X\b + \eps$ with $\E[\eps]
    = \zero_n$ and $\Cov(\eps) = \sigma^2 \I_n$ and with $\X$ of full
    column rank $(p+1)$.
  \begin{enumerate}
\item Consider estimation of $\b$ using quadratic loss $(\b -
  \a)^T(\b - \a)$ for some estimator $\a$.  Find the expected quadratic
  loss if we use the MLE $\bhat$ for $\a$. Simplify the expression
  as a function of the eigenvalues of $\X^T\X$.   What happens as the
  smallest eigenvalue goes to 0?
\item Consider estimation $\mub$'s at the observed data points
  $\X$.  Find the expected  quadratic loss   $\E[(\mub -
  \X\bhat)^T(\mub - \X\bhat)]$.  What happens as the  smallest eigenvalue of $\X^T\X$ goes to 0?
\item Consider predicting $\Y_*$'s at the observed data points $\X$
  where $\Y_*$ is independent of $\Y$.  Find the expected quadratic
  loss $\E[(\Y_* - \X\bhat)^T(\Y_* - \X\bhat)]$.  What happens as the
  smallest eigenvalue of $\X^T\X$ goes to 0?
\item Consider predicting $\Y_*$'s at new points $\X_*$ with
  $\E[\X_*^T\X_*] = \I_p$.  Find the expected quadratic loss
  $\E[(\Y_* - \X_*\bhat)^T(\Y_* - \X_*\bhat)]$.  What
  happens as the smallest eigenvalue of $\X^T\X$ goes to 0?  (If
  $E[\X_*^T\X_*] = \Sigmab > 0$ does that change the result)
\item Comment on the difference in estimation, prediction at observed
  data and prediction at new data as $\X$ becomes non-full rank.
  Which is the most stable?  Which is the least?
  \end{enumerate}
\end{enumerate}

\end{document}
